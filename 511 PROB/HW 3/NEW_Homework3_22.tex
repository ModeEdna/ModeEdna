% Options for packages loaded elsewhere
\PassOptionsToPackage{unicode}{hyperref}
\PassOptionsToPackage{hyphens}{url}
%
\documentclass[
]{article}
\usepackage{amsmath,amssymb}
\usepackage{lmodern}
\usepackage{iftex}
\ifPDFTeX
  \usepackage[T1]{fontenc}
  \usepackage[utf8]{inputenc}
  \usepackage{textcomp} % provide euro and other symbols
\else % if luatex or xetex
  \usepackage{unicode-math}
  \defaultfontfeatures{Scale=MatchLowercase}
  \defaultfontfeatures[\rmfamily]{Ligatures=TeX,Scale=1}
\fi
% Use upquote if available, for straight quotes in verbatim environments
\IfFileExists{upquote.sty}{\usepackage{upquote}}{}
\IfFileExists{microtype.sty}{% use microtype if available
  \usepackage[]{microtype}
  \UseMicrotypeSet[protrusion]{basicmath} % disable protrusion for tt fonts
}{}
\makeatletter
\@ifundefined{KOMAClassName}{% if non-KOMA class
  \IfFileExists{parskip.sty}{%
    \usepackage{parskip}
  }{% else
    \setlength{\parindent}{0pt}
    \setlength{\parskip}{6pt plus 2pt minus 1pt}}
}{% if KOMA class
  \KOMAoptions{parskip=half}}
\makeatother
\usepackage{xcolor}
\usepackage[margin=1in]{geometry}
\usepackage{color}
\usepackage{fancyvrb}
\newcommand{\VerbBar}{|}
\newcommand{\VERB}{\Verb[commandchars=\\\{\}]}
\DefineVerbatimEnvironment{Highlighting}{Verbatim}{commandchars=\\\{\}}
% Add ',fontsize=\small' for more characters per line
\usepackage{framed}
\definecolor{shadecolor}{RGB}{248,248,248}
\newenvironment{Shaded}{\begin{snugshade}}{\end{snugshade}}
\newcommand{\AlertTok}[1]{\textcolor[rgb]{0.94,0.16,0.16}{#1}}
\newcommand{\AnnotationTok}[1]{\textcolor[rgb]{0.56,0.35,0.01}{\textbf{\textit{#1}}}}
\newcommand{\AttributeTok}[1]{\textcolor[rgb]{0.77,0.63,0.00}{#1}}
\newcommand{\BaseNTok}[1]{\textcolor[rgb]{0.00,0.00,0.81}{#1}}
\newcommand{\BuiltInTok}[1]{#1}
\newcommand{\CharTok}[1]{\textcolor[rgb]{0.31,0.60,0.02}{#1}}
\newcommand{\CommentTok}[1]{\textcolor[rgb]{0.56,0.35,0.01}{\textit{#1}}}
\newcommand{\CommentVarTok}[1]{\textcolor[rgb]{0.56,0.35,0.01}{\textbf{\textit{#1}}}}
\newcommand{\ConstantTok}[1]{\textcolor[rgb]{0.00,0.00,0.00}{#1}}
\newcommand{\ControlFlowTok}[1]{\textcolor[rgb]{0.13,0.29,0.53}{\textbf{#1}}}
\newcommand{\DataTypeTok}[1]{\textcolor[rgb]{0.13,0.29,0.53}{#1}}
\newcommand{\DecValTok}[1]{\textcolor[rgb]{0.00,0.00,0.81}{#1}}
\newcommand{\DocumentationTok}[1]{\textcolor[rgb]{0.56,0.35,0.01}{\textbf{\textit{#1}}}}
\newcommand{\ErrorTok}[1]{\textcolor[rgb]{0.64,0.00,0.00}{\textbf{#1}}}
\newcommand{\ExtensionTok}[1]{#1}
\newcommand{\FloatTok}[1]{\textcolor[rgb]{0.00,0.00,0.81}{#1}}
\newcommand{\FunctionTok}[1]{\textcolor[rgb]{0.00,0.00,0.00}{#1}}
\newcommand{\ImportTok}[1]{#1}
\newcommand{\InformationTok}[1]{\textcolor[rgb]{0.56,0.35,0.01}{\textbf{\textit{#1}}}}
\newcommand{\KeywordTok}[1]{\textcolor[rgb]{0.13,0.29,0.53}{\textbf{#1}}}
\newcommand{\NormalTok}[1]{#1}
\newcommand{\OperatorTok}[1]{\textcolor[rgb]{0.81,0.36,0.00}{\textbf{#1}}}
\newcommand{\OtherTok}[1]{\textcolor[rgb]{0.56,0.35,0.01}{#1}}
\newcommand{\PreprocessorTok}[1]{\textcolor[rgb]{0.56,0.35,0.01}{\textit{#1}}}
\newcommand{\RegionMarkerTok}[1]{#1}
\newcommand{\SpecialCharTok}[1]{\textcolor[rgb]{0.00,0.00,0.00}{#1}}
\newcommand{\SpecialStringTok}[1]{\textcolor[rgb]{0.31,0.60,0.02}{#1}}
\newcommand{\StringTok}[1]{\textcolor[rgb]{0.31,0.60,0.02}{#1}}
\newcommand{\VariableTok}[1]{\textcolor[rgb]{0.00,0.00,0.00}{#1}}
\newcommand{\VerbatimStringTok}[1]{\textcolor[rgb]{0.31,0.60,0.02}{#1}}
\newcommand{\WarningTok}[1]{\textcolor[rgb]{0.56,0.35,0.01}{\textbf{\textit{#1}}}}
\usepackage{graphicx}
\makeatletter
\def\maxwidth{\ifdim\Gin@nat@width>\linewidth\linewidth\else\Gin@nat@width\fi}
\def\maxheight{\ifdim\Gin@nat@height>\textheight\textheight\else\Gin@nat@height\fi}
\makeatother
% Scale images if necessary, so that they will not overflow the page
% margins by default, and it is still possible to overwrite the defaults
% using explicit options in \includegraphics[width, height, ...]{}
\setkeys{Gin}{width=\maxwidth,height=\maxheight,keepaspectratio}
% Set default figure placement to htbp
\makeatletter
\def\fps@figure{htbp}
\makeatother
\setlength{\emergencystretch}{3em} % prevent overfull lines
\providecommand{\tightlist}{%
  \setlength{\itemsep}{0pt}\setlength{\parskip}{0pt}}
\setcounter{secnumdepth}{-\maxdimen} % remove section numbering
\ifLuaTeX
  \usepackage{selnolig}  % disable illegal ligatures
\fi
\IfFileExists{bookmark.sty}{\usepackage{bookmark}}{\usepackage{hyperref}}
\IfFileExists{xurl.sty}{\usepackage{xurl}}{} % add URL line breaks if available
\urlstyle{same} % disable monospaced font for URLs
\hypersetup{
  pdftitle={Homework 3},
  pdfauthor={Eduardo Armenta},
  hidelinks,
  pdfcreator={LaTeX via pandoc}}

\title{Homework 3}
\author{Eduardo Armenta}
\date{}

\begin{document}
\maketitle

\hypertarget{problem-1-10-points}{%
\subsection{Problem 1 (10 Points)}\label{problem-1-10-points}}

Consider a room that is paved with \(n \times n\) square tiles which are
labeled from 1 to \(n^2\) in some order. A frog performs a random walk
by hopping from one tile to a randomly chosen adjacent tile in each time
step. All adjacent tiles are chosen with the same probability. The frog
can never hop into a wall of the room.

\textbf{True or not true:} The transition matrix for this random walk is
symmetric, that is, it satisfies
\(P(X_{i+1} = k| X_i = j) = P(X_{i+1} = j|X_i = k)\) for all \(i\) and
all possible states \(1 \le j, \, k \le n^2\). Explain your answer.

It is not symmetric. If you take the example of a 3x3 matrix, and the
tiles are numbered from left to right, bottom to top, the 1st tile has a
probability of 1/2 to go to tile 2, while tile 2 has a probability of
1/3 to go to tile 1. For it to be symmetric, these probabilities would
have to be the same.

\hypertarget{problem-2-30-points}{%
\subsection{Problem 2 (30 Points)}\label{problem-2-30-points}}

Let \(X \sim B(80, .2)\) and \(Y \sim B(100, .7)\) be independent
binomial random variables. Let \(Z = X + Y\) . Find the following
conditional quantities, using R simulations:

\begin{Shaded}
\begin{Highlighting}[]
\NormalTok{df }\OtherTok{\textless{}{-}} \FunctionTok{data.frame}\NormalTok{(}\AttributeTok{x=}\FunctionTok{rbinom}\NormalTok{(}\DecValTok{10000}\NormalTok{, }\DecValTok{80}\NormalTok{, .}\DecValTok{2}\NormalTok{), }\AttributeTok{y=}\FunctionTok{rbinom}\NormalTok{(}\DecValTok{10000}\NormalTok{, }\DecValTok{100}\NormalTok{, .}\DecValTok{7}\NormalTok{))}
\NormalTok{df}\SpecialCharTok{$}\NormalTok{z }\OtherTok{\textless{}{-}}\NormalTok{ df}\SpecialCharTok{$}\NormalTok{x }\SpecialCharTok{+}\NormalTok{ df}\SpecialCharTok{$}\NormalTok{y}
\end{Highlighting}
\end{Shaded}

\begin{enumerate}
\def\labelenumi{\alph{enumi})}
\tightlist
\item
  \(P(X < 12|X < 18)\) and \(E(X|X < 18)\) (6 points)
\end{enumerate}

\begin{Shaded}
\begin{Highlighting}[]
\CommentTok{\# probability of x being less than 12 if x is less than 18}
\FunctionTok{length}\NormalTok{(df}\SpecialCharTok{$}\NormalTok{x[df}\SpecialCharTok{$}\NormalTok{x}\SpecialCharTok{\textless{}}\DecValTok{12}\NormalTok{])}\SpecialCharTok{/}\FunctionTok{length}\NormalTok{(df}\SpecialCharTok{$}\NormalTok{x[df}\SpecialCharTok{$}\NormalTok{x}\SpecialCharTok{\textless{}}\DecValTok{18}\NormalTok{])}
\end{Highlighting}
\end{Shaded}

\begin{verbatim}
## [1] 0.1468918
\end{verbatim}

\begin{Shaded}
\begin{Highlighting}[]
\CommentTok{\# expected value}
\FunctionTok{mean}\NormalTok{(df}\SpecialCharTok{$}\NormalTok{x[df}\SpecialCharTok{$}\NormalTok{x}\SpecialCharTok{\textless{}}\DecValTok{18}\NormalTok{])}
\end{Highlighting}
\end{Shaded}

\begin{verbatim}
## [1] 14.0514
\end{verbatim}

\begin{enumerate}
\def\labelenumi{\alph{enumi})}
\setcounter{enumi}{1}
\tightlist
\item
  the cumulative distribution function of \(X|(12 \le X \le 20)\) (plot
  of the ecdf) (6 points)
\end{enumerate}

\begin{Shaded}
\begin{Highlighting}[]
\FunctionTok{plot.ecdf}\NormalTok{(df}\SpecialCharTok{$}\NormalTok{x[df}\SpecialCharTok{$}\NormalTok{x}\SpecialCharTok{\textless{}=}\DecValTok{20} \SpecialCharTok{\&}\NormalTok{ df}\SpecialCharTok{$}\NormalTok{x}\SpecialCharTok{\textgreater{}=}\DecValTok{12}\NormalTok{])}
\end{Highlighting}
\end{Shaded}

\includegraphics{NEW_Homework3_22_files/figure-latex/unnamed-chunk-3-1.pdf}

\begin{enumerate}
\def\labelenumi{\alph{enumi})}
\setcounter{enumi}{2}
\tightlist
\item
  the cumulative distribution function of \(X|Z = 90\) ( plot of the
  ecdf )(6 points)
\end{enumerate}

\begin{Shaded}
\begin{Highlighting}[]
\FunctionTok{plot.ecdf}\NormalTok{(df}\SpecialCharTok{$}\NormalTok{x[df}\SpecialCharTok{$}\NormalTok{z}\SpecialCharTok{==}\DecValTok{90}\NormalTok{])}
\end{Highlighting}
\end{Shaded}

\includegraphics{NEW_Homework3_22_files/figure-latex/unnamed-chunk-4-1.pdf}
d) \(E(Z|X = k)\) for \(k = 10, 15, 20\) . (6 points)

\begin{Shaded}
\begin{Highlighting}[]
\FunctionTok{mean}\NormalTok{(df}\SpecialCharTok{$}\NormalTok{z[df}\SpecialCharTok{$}\NormalTok{x}\SpecialCharTok{==}\DecValTok{10} \SpecialCharTok{|}\NormalTok{ df}\SpecialCharTok{$}\NormalTok{x}\SpecialCharTok{==}\DecValTok{15} \SpecialCharTok{|}\NormalTok{ df}\SpecialCharTok{$}\NormalTok{x}\SpecialCharTok{==}\DecValTok{20}\NormalTok{])}
\end{Highlighting}
\end{Shaded}

\begin{verbatim}
## [1] 85.90436
\end{verbatim}

\begin{enumerate}
\def\labelenumi{\alph{enumi})}
\setcounter{enumi}{4}
\tightlist
\item
  \(E(X|Z = k)\) for \(k = 80, 90, 100\) . (6 points)
\end{enumerate}

\begin{Shaded}
\begin{Highlighting}[]
\FunctionTok{mean}\NormalTok{(df}\SpecialCharTok{$}\NormalTok{x[df}\SpecialCharTok{$}\NormalTok{z}\SpecialCharTok{==}\DecValTok{80} \SpecialCharTok{|}\NormalTok{ df}\SpecialCharTok{$}\NormalTok{z}\SpecialCharTok{==}\DecValTok{90} \SpecialCharTok{|}\NormalTok{ df}\SpecialCharTok{$}\NormalTok{z}\SpecialCharTok{==}\DecValTok{100}\NormalTok{])}
\end{Highlighting}
\end{Shaded}

\begin{verbatim}
## [1] 16.11394
\end{verbatim}

\hypertarget{problem-3-20-points}{%
\subsection{Problem 3 (20 Points)}\label{problem-3-20-points}}

Suppose \(X\) has an exponential distribution with parameter
\(\lambda = 1\) and \(Y|X = x\) has a Poisson distribution with
parameter \(x\).

\begin{enumerate}
\def\labelenumi{\alph{enumi})}
\item
  Generate at least 1000 random samples from the marginal distribution
  of \(X\) and make a probability histogram. (4 Points)
\item
  Generate at least 1000 random samples from the conditional
  distribution of \(Y|X = 1.5\) and make a probability histogram. (6
  Points)
\item
  Generate at least 1000 random samples from the marginal distribution
  of \(Y\) and make a probability histogram. (4 Points)
\item
  Generate at least 1000 random samples from the conditional
  distribution of \(X|Y=2\) and make a probability histogram. (6 Points)
\end{enumerate}

\hypertarget{problem-4-20-points}{%
\subsection{Problem 4 (20 Points)}\label{problem-4-20-points}}

Suppose \(X\) and \(Y\) have independent standard normal distributions.
Make at least 1,000 random samples from \(Z\), defined as
\(Z = Y|(X + Y \ge 1)\). Do you think that \(Z\) has a normal
distribution? What are its approximate mean and standard deviation?

\begin{Shaded}
\begin{Highlighting}[]
\NormalTok{df }\OtherTok{\textless{}{-}} \FunctionTok{data.frame}\NormalTok{(}\AttributeTok{x=}\FunctionTok{rnorm}\NormalTok{(}\DecValTok{10000}\NormalTok{), }\AttributeTok{y=}\FunctionTok{rnorm}\NormalTok{(}\DecValTok{10000}\NormalTok{))}
\NormalTok{df}\SpecialCharTok{$}\NormalTok{z }\OtherTok{\textless{}{-}}\NormalTok{ df}\SpecialCharTok{$}\NormalTok{x}\SpecialCharTok{+}\NormalTok{df}\SpecialCharTok{$}\NormalTok{y}
\NormalTok{Z }\OtherTok{\textless{}{-}}\NormalTok{ df[df}\SpecialCharTok{$}\NormalTok{z}\SpecialCharTok{\textgreater{}=}\DecValTok{1}\NormalTok{,]}
\NormalTok{sampZ }\OtherTok{\textless{}{-}} \FunctionTok{sample}\NormalTok{(Z}\SpecialCharTok{$}\NormalTok{y,}\DecValTok{1000}\NormalTok{)}
\FunctionTok{mean}\NormalTok{(sampZ)}
\end{Highlighting}
\end{Shaded}

\begin{verbatim}
## [1] 0.9591247
\end{verbatim}

\begin{Shaded}
\begin{Highlighting}[]
\FunctionTok{sd}\NormalTok{(sampZ)}
\end{Highlighting}
\end{Shaded}

\begin{verbatim}
## [1] 0.7869094
\end{verbatim}

\begin{Shaded}
\begin{Highlighting}[]
\FunctionTok{hist}\NormalTok{(sampZ)}
\end{Highlighting}
\end{Shaded}

\includegraphics{NEW_Homework3_22_files/figure-latex/unnamed-chunk-7-1.pdf}

\hypertarget{problem-5-submit-only-part-1-and-part-4-practice-and-discuss-the-parts-2-and-3-with-classmates-20-points}{%
\subsection{\texorpdfstring{Problem 5 (Submit only \emph{\emph{part 1}}
and \emph{\emph{part 4}}; \emph{practice and discuss the parts 2 and 3
with classmates}) (20
Points)}{Problem 5 (Submit only part 1 and part 4; practice and discuss the parts 2 and 3 with classmates) (20 Points)}}\label{problem-5-submit-only-part-1-and-part-4-practice-and-discuss-the-parts-2-and-3-with-classmates-20-points}}

\textbf{Mixtures.} Let \(Y_1\) and \(Y_2\) be two random variables which
have the same range \(R\), and let \(w_1, w_2\) probabilities with
\(w_1 + w_2 = 1\). Then the mixture \(Y\) of \(Y_1\) and \(Y_2\) is
defined as follows:

\begin{itemize}
\item
  Select \(X \in \{1,2\}\) at random, with
  \(P(X = 1) = w_1, \, P(X = 2) = w_2\).
\item
  If \(X = 1\), draw a sample \(Y_1\) and set \(Y = Y_1\). Otherwise,
  draw a sample \(Y_2\) and set \(Y = Y_2\).
\end{itemize}

\smallskip

\begin{enumerate}
\def\labelenumi{\arabic{enumi}.}
\item
  Suppose \(E(Y_1) = \mu_1\) and \(E(Y_2) = \mu_2\). What is
  \(E(Y|X = 1)\)? What is \(E(Y|X=2)\)? Use this to show that
  \(E(Y) = w_1 \mu_1 + w_2 \mu_2\). (5 Points)
\item
  Suppose \(var(Y_1) = \sigma_1^2\) and \(var(Y_2) = \sigma_2^2\).
  Explain why \(E(Y^2|X = 1) = \sigma_1^2 + \mu_1^2\) and
  \(E(Y^2|X=2) = \sigma_2^2 + \mu_2^2\). Use this to find a formula for
  \(E(Y^2)\).
\item
  Use the results of a) and b) to find a formula for \(var(Y)\).
\item
  Generate a sample of size 10,000 from
  \(Y_1 \sim N(-2,1), \, Y_2 \sim N(2,2), \, w_1 = \frac{1}{3}, \, w_2 = \frac{2}{3}\)
  and make a probability histogram. Clearly this is not a normal
  distribution, and a mixture is not a sum! (15 Points)
\end{enumerate}

\hypertarget{bonus}{%
\subsection{BONUS}\label{bonus}}

Bob's preferred bet in American roulette consists in betting \$1 on
black numbers and simultaneously \$2 on even numbers (see the roulette
board in the course slides). Find all possible outcomes of a single game
and their probabilities, that is, find the probability distribution of
the outcome of a single bet. Then compute its expected value.

(Hint: Since the question is asking about a single bet, you can
calculate this by hand)

Possible outcome probs: Black: 18/38 (would win \$2) Even: 18/38 (would
win \$4) Loss: 20/38 (would lose \$3) Black and Even: 10/38 (would win
\$6)

We must take 10 away from black and even since black and even together
has 10 possible outcomes. Calculate expected winnings on each round:
2\emph{(8/38) + 4}(8/38) + 6*(10/38) = 2.84 are the expected winnings
for each round

\end{document}
