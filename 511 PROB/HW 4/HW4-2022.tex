% Options for packages loaded elsewhere
\PassOptionsToPackage{unicode}{hyperref}
\PassOptionsToPackage{hyphens}{url}
%
\documentclass[
]{article}
\usepackage{amsmath,amssymb}
\usepackage{lmodern}
\usepackage{iftex}
\ifPDFTeX
  \usepackage[T1]{fontenc}
  \usepackage[utf8]{inputenc}
  \usepackage{textcomp} % provide euro and other symbols
\else % if luatex or xetex
  \usepackage{unicode-math}
  \defaultfontfeatures{Scale=MatchLowercase}
  \defaultfontfeatures[\rmfamily]{Ligatures=TeX,Scale=1}
\fi
% Use upquote if available, for straight quotes in verbatim environments
\IfFileExists{upquote.sty}{\usepackage{upquote}}{}
\IfFileExists{microtype.sty}{% use microtype if available
  \usepackage[]{microtype}
  \UseMicrotypeSet[protrusion]{basicmath} % disable protrusion for tt fonts
}{}
\makeatletter
\@ifundefined{KOMAClassName}{% if non-KOMA class
  \IfFileExists{parskip.sty}{%
    \usepackage{parskip}
  }{% else
    \setlength{\parindent}{0pt}
    \setlength{\parskip}{6pt plus 2pt minus 1pt}}
}{% if KOMA class
  \KOMAoptions{parskip=half}}
\makeatother
\usepackage{xcolor}
\usepackage[margin=1in]{geometry}
\usepackage{color}
\usepackage{fancyvrb}
\newcommand{\VerbBar}{|}
\newcommand{\VERB}{\Verb[commandchars=\\\{\}]}
\DefineVerbatimEnvironment{Highlighting}{Verbatim}{commandchars=\\\{\}}
% Add ',fontsize=\small' for more characters per line
\usepackage{framed}
\definecolor{shadecolor}{RGB}{248,248,248}
\newenvironment{Shaded}{\begin{snugshade}}{\end{snugshade}}
\newcommand{\AlertTok}[1]{\textcolor[rgb]{0.94,0.16,0.16}{#1}}
\newcommand{\AnnotationTok}[1]{\textcolor[rgb]{0.56,0.35,0.01}{\textbf{\textit{#1}}}}
\newcommand{\AttributeTok}[1]{\textcolor[rgb]{0.77,0.63,0.00}{#1}}
\newcommand{\BaseNTok}[1]{\textcolor[rgb]{0.00,0.00,0.81}{#1}}
\newcommand{\BuiltInTok}[1]{#1}
\newcommand{\CharTok}[1]{\textcolor[rgb]{0.31,0.60,0.02}{#1}}
\newcommand{\CommentTok}[1]{\textcolor[rgb]{0.56,0.35,0.01}{\textit{#1}}}
\newcommand{\CommentVarTok}[1]{\textcolor[rgb]{0.56,0.35,0.01}{\textbf{\textit{#1}}}}
\newcommand{\ConstantTok}[1]{\textcolor[rgb]{0.00,0.00,0.00}{#1}}
\newcommand{\ControlFlowTok}[1]{\textcolor[rgb]{0.13,0.29,0.53}{\textbf{#1}}}
\newcommand{\DataTypeTok}[1]{\textcolor[rgb]{0.13,0.29,0.53}{#1}}
\newcommand{\DecValTok}[1]{\textcolor[rgb]{0.00,0.00,0.81}{#1}}
\newcommand{\DocumentationTok}[1]{\textcolor[rgb]{0.56,0.35,0.01}{\textbf{\textit{#1}}}}
\newcommand{\ErrorTok}[1]{\textcolor[rgb]{0.64,0.00,0.00}{\textbf{#1}}}
\newcommand{\ExtensionTok}[1]{#1}
\newcommand{\FloatTok}[1]{\textcolor[rgb]{0.00,0.00,0.81}{#1}}
\newcommand{\FunctionTok}[1]{\textcolor[rgb]{0.00,0.00,0.00}{#1}}
\newcommand{\ImportTok}[1]{#1}
\newcommand{\InformationTok}[1]{\textcolor[rgb]{0.56,0.35,0.01}{\textbf{\textit{#1}}}}
\newcommand{\KeywordTok}[1]{\textcolor[rgb]{0.13,0.29,0.53}{\textbf{#1}}}
\newcommand{\NormalTok}[1]{#1}
\newcommand{\OperatorTok}[1]{\textcolor[rgb]{0.81,0.36,0.00}{\textbf{#1}}}
\newcommand{\OtherTok}[1]{\textcolor[rgb]{0.56,0.35,0.01}{#1}}
\newcommand{\PreprocessorTok}[1]{\textcolor[rgb]{0.56,0.35,0.01}{\textit{#1}}}
\newcommand{\RegionMarkerTok}[1]{#1}
\newcommand{\SpecialCharTok}[1]{\textcolor[rgb]{0.00,0.00,0.00}{#1}}
\newcommand{\SpecialStringTok}[1]{\textcolor[rgb]{0.31,0.60,0.02}{#1}}
\newcommand{\StringTok}[1]{\textcolor[rgb]{0.31,0.60,0.02}{#1}}
\newcommand{\VariableTok}[1]{\textcolor[rgb]{0.00,0.00,0.00}{#1}}
\newcommand{\VerbatimStringTok}[1]{\textcolor[rgb]{0.31,0.60,0.02}{#1}}
\newcommand{\WarningTok}[1]{\textcolor[rgb]{0.56,0.35,0.01}{\textbf{\textit{#1}}}}
\usepackage{graphicx}
\makeatletter
\def\maxwidth{\ifdim\Gin@nat@width>\linewidth\linewidth\else\Gin@nat@width\fi}
\def\maxheight{\ifdim\Gin@nat@height>\textheight\textheight\else\Gin@nat@height\fi}
\makeatother
% Scale images if necessary, so that they will not overflow the page
% margins by default, and it is still possible to overwrite the defaults
% using explicit options in \includegraphics[width, height, ...]{}
\setkeys{Gin}{width=\maxwidth,height=\maxheight,keepaspectratio}
% Set default figure placement to htbp
\makeatletter
\def\fps@figure{htbp}
\makeatother
\setlength{\emergencystretch}{3em} % prevent overfull lines
\providecommand{\tightlist}{%
  \setlength{\itemsep}{0pt}\setlength{\parskip}{0pt}}
\setcounter{secnumdepth}{-\maxdimen} % remove section numbering
\ifLuaTeX
  \usepackage{selnolig}  % disable illegal ligatures
\fi
\IfFileExists{bookmark.sty}{\usepackage{bookmark}}{\usepackage{hyperref}}
\IfFileExists{xurl.sty}{\usepackage{xurl}}{} % add URL line breaks if available
\urlstyle{same} % disable monospaced font for URLs
\hypersetup{
  pdftitle={Homework 4 Key},
  pdfauthor={Dr.~Purna Gamage},
  hidelinks,
  pdfcreator={LaTeX via pandoc}}

\title{Homework 4 Key}
\author{Dr.~Purna Gamage}
\date{10/7/2020}

\begin{document}
\maketitle

\hypertarget{question-1-16-points}{%
\subsection{Question 1: 16 points}\label{question-1-16-points}}

Assume there are a total of 20 congressional seats up for election
across the Unites States which has a multinomial distribution. We will
assume that for every seat there are 3 candidates running, 1 from the
Democratic party, 1 from the Republican party and 1 from an Independent
party. We will also assume that for every seat there is a 45\% chance
the Democratic candidate will win, a 45\% the Republican candidate will
win and a 10\% chance the Independent will win.

Let Democrat be denoted as \(D\), Republican as \(R\) and Independent as
\(I\)

Use a simulation with \texttt{rmultinom} to show that
\(P(D = 9, R = 9, I = 2) \approx 0.0529\). Confirm your results using
\texttt{dmultinom}.

\hypertarget{question-2-18-points}{%
\subsection{Question 2: 18 points}\label{question-2-18-points}}

Suppose \(X = (X_1,X_2,X_3)\) has a multinomial distribution with size
\(n = 10\) and probabilities \(p_1 = .3, p_2 = .4, p_3 = .3\). Use a
simulation with \texttt{sample} (not \texttt{rmultinom}) to show that
\(P(X_1 = 3, X_2 = 4, X_3 = 3) \approx 0.0784\). Confirm your results
using \textbf{dmultinom}.

\hypertarget{question-3-14-points}{%
\subsection{Question 3: 14 points}\label{question-3-14-points}}

Let \(X_1, \dots, X_{12}\) be a random sample of size 12 from the
\(U(0,1)\) distribution. Explain why
\(Z = X_1 + X_2 + \dots + X_{12} - 6\) has an approximate standard
normal distribution. You can either prove this theoretically by using
CLT, or can use a simulation.
\textit{You will have to find or look up the variance of a single $X_i$.}

\hypertarget{question-4-20-points}{%
\subsection{Question 4: 20 points}\label{question-4-20-points}}

Problem 4.4 \#14 a (8) and b(12) in Chihara/Hesterberg.

\includegraphics{14.png}

Hint:

\includegraphics{A.2.png} \includegraphics{A.10.png}

\hypertarget{question-5-32-points-8-for-each}{%
\subsection{Question 5 : 32 points, 8 for
each}\label{question-5-32-points-8-for-each}}

Problem 4.4 \#18 in Chihara/Hesterberg.

A. Simulate the sampling distribution.

\begin{Shaded}
\begin{Highlighting}[]
\CommentTok{\# get the 30 numbers and their mean}
\NormalTok{exp}\FloatTok{.1} \OtherTok{\textless{}{-}} \FunctionTok{rexp}\NormalTok{(}\DecValTok{30}\NormalTok{,}\DecValTok{1}\SpecialCharTok{/}\DecValTok{3}\NormalTok{)}
\NormalTok{X.hat }\OtherTok{\textless{}{-}} \FunctionTok{mean}\NormalTok{(exp}\FloatTok{.1}\NormalTok{)}

\CommentTok{\# get length of sample and bootstrap it}
\NormalTok{n.exp }\OtherTok{\textless{}{-}} \FunctionTok{length}\NormalTok{(exp}\FloatTok{.1}\NormalTok{)}
\NormalTok{exp.boot }\OtherTok{\textless{}{-}} \FunctionTok{replicate}\NormalTok{(}\DecValTok{10000}\NormalTok{,}\FunctionTok{mean}\NormalTok{(}\FunctionTok{sample}\NormalTok{(exp}\FloatTok{.1}\NormalTok{, n.exp, }\AttributeTok{replace=}\ConstantTok{TRUE}\NormalTok{)))}
\FunctionTok{hist}\NormalTok{(exp.boot, }\AttributeTok{prob=}\NormalTok{T)}
\end{Highlighting}
\end{Shaded}

\includegraphics{HW4-2022_files/figure-latex/unnamed-chunk-1-1.pdf}

\hypertarget{problem-6-bonus-10-points}{%
\subsection{Problem 6 (Bonus, 10
Points)}\label{problem-6-bonus-10-points}}

Let \(X_1, \dots, X_n\) be a random sample of size \(n\) from a
\(U(0,a)\) distribution, where \(a > 0\). Find
\(E(X_1+X_2+ \dots + X_n)\). and find the approximate distribution of
the sample mean, if \(n\) is large.

\end{document}
